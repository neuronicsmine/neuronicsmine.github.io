\documentclass[12pt]{article}



\oddsidemargin -0.5cm \evensidemargin -0.5cm 
\marginparwidth 40pt \marginparsep 10pt
\topmargin -35pt \headsep 35pt
\textheight 23cm \textwidth 17cm
\pagestyle{empty}

\begin{document}

\begin{large}
\begin{center}
{\bf Pratiksha Postdoctoral Fellowship}\\
at {\bf Indian Institute of Science, Bangalore, India}
\end{center}
\end{large}


\vspace*{0.75cm}
\noindent
The Indian Institute of Science (IISc) invites applications for {\bf Pratiksha Postdoctoral Fellowship}. This fellowship is supported by an endowment by Pratiksha Trust set up by Mr. Kris Gopalakrishnan and Mrs. Sudha Gopalakrishnan. The main objective of the fellowship is to promote interdisciplinary research that  interfaces computer science with neurobiology.  \\

\noindent
Computational approaches to understanding brain function form an important and growing area of interdisciplinary research. Gaining a detailed understanding of human brain has been termed one of the grandest challenges of the 21st century. The grandness of the challenge and the requirement of diverse forms of expertise necessitate synergistic interactions among neurobiologists, computer scientists and electrical engineers. There are about 30 faculty members from different departments at IISc who are actively pursuing research on various aspects of this interdisciplinary area. \\



\noindent
{\bf Eligibility and emoluments}: The applicants should have a Ph.D. degree and should have good research accomplishments in the areas of Electrical, Electronics and Computer Sciences or Neurobiology with broad relevance to Brain Function, Data Science, Machine Learning or Neuromorphic Computing. It is not necessary that the applicant should be an Indian citizen. The postdoctoral fellows are paid a consolidated salary of Rs. 1,00,000 per month and are eligible for a contingency grant of Rs. 2,00,000 per year. The fellowship is for a period of one year which can be extended by one more year based on an evaluation at the end of the first year. \\

\noindent
{\bf Application procedure}: Applicants should identify a faculty member from IISc who is willing to be their mentor (For applicants who obtained their Ph.D. from IISc, the mentor should not be their Ph.D. advisor). Each application should contain a detailed CV, copies of three best papers and a letter of support from their mentor at IISc. The application should also contain names and full addresses of at least three senior researchers who are familiar with the applicant's work, from whom letters of reference can be requested. \\

\noindent
Applications should be sent by e-mail to \\
{\bf PratikshaPostdoctoralFellowship.ee@iisc.ac.in}

\end{document}
